\RequirePackage{plautopatch}
\RequirePackage[l2tabu, orthodox]{nag}

\documentclass[platex,dvipdfmx]{jlreq}			% for platex
% \documentclass[uplatex,dvipdfmx]{jlreq}		% for uplatex
\usepackage{graphicx}
\usepackage{bxtexlogo}
\usepackage{plext}
\usepackage{amsmath,amssymb}
\usepackage{ascmac}
\usepackage{fancybox}
\title{線型計画}

\author{T-N}
\date{\today}
\begin{document}
\maketitle
\section*{3章}

\subsection*{3-1.線型計画問題の標準形}
\noindent
下記のような線形の等式制約と非負制約の下で目的関数を最適化する問題のことを$\textbf{線型計画問題の標準形}$と呼ぶ。\\\\
\large
$ w=c_1x_1+c_2x_2+c_3x_3+\cdots+c_nx_n $ $\quad$←最小化したい目的関数\\\\
$ a_{11}x_1+a_{12}x_2+\cdots+a_{1n}x_n =b_1 $ $\quad$ ←制約条件\\
$ a_{21}x_1+a_{22}x_2+\cdots+a_{2n}x_n =b_2$\\
\hspace{3cm} $\vdots$\\
$ a_{m1}x_1+a_{m2}x_2+\cdots+a_{mn}x_n =b_m$\\
\normalsize
 $x_i\geqq0$\\

\noindent
下記のようにベクトルを定義すれば、線型計画標準形は違う書き方もできる。


$x= \begin{pmatrix}
x_1\\
x_2\\
\vdots\\
x_n
\end{pmatrix}
$, \quad
$c= \begin{pmatrix}
c_1\\
c_2\\
\vdots\\
c_n
\end{pmatrix}
$,\quad
$b= \begin{pmatrix}
b_1\\
b_2\\
\vdots\\
b_n
\end{pmatrix}
$,\quad
$A= \begin{pmatrix}
a_{11} & a_{12} &\cdots & a_{1n}\\
a_{21} & a_{22} &\cdots & a_{2n}\\
\vdots & \vdots &  \ddots &   \vdots \\
a_{m1} & a_{m2}& \cdots &  a_{mn}
\end{pmatrix}
$\\

\begin{itembox}[l]{線型計画問題の標準形}
最小化→$ w=c^Tx $\\
制約条件→$ Ax=b \quad (x_i\geqq0)$
\end{itembox}

\subsection*{3-2.線型計画標準形の拡張}

世の中には最大化を目的とする問題もある。(利益の最大化など)\\
これもうまく式を変形すれば線型計画の標準形であらわすことができる。\\
ある制約の下で$f(x)$を最大化したい時には$-f(x)$を最小化すればいい。

次に制約条件が不等号であらわされる場合について考えてみよう。
そのような場合には$\textbf{スラック変数}$を導入すればよい。このスラック変数とはいったい何なのか?具体例も見てみよう。\\
$g(n) \leqq b $
のような不等号で表現できる制約条件があるとする。\\
ここで、非負のスラック変数$x$を導入すると制約条件を$g(n)+x = b$と書き直すことができる。これがスラック変数の効果である。
不等号の向きが反対の場合、つまり $g(n) \geqq b $  の場合は両辺を-1倍してからスラック変数を導入すればいい。
つまり $g(n) \geqq b$ という条件があった場合にスラック変数を導入すると、 $-g(x)+x = -b$ というような制約条件に書き換えることができる。\\
(左辺からスラック変数を減じて$g(n)-x=b$と書き直すのもアリ)

最後に自由変数について見ていこう。
制約条件の変数に非負の制約がない場合には、変数の個数を2倍にする代わりに非負制約を課すことができる。
つまり、非負制約のない変数$x$が存在すると仮定すると$x = x^+ - x^- (x^+ \geqq 0 , \: x^- \geqq 0) $と表すことで、変数の個数を2倍にする代わりに非負制約を導入することができるという事である。\\



\subsection*{3-3.実行可能解と基底解}

\begin{itembox}[l]{線型計画問題の標準形}
最小化→$ w=c^Tx $\\
制約条件→$ Ax=b \quad (x_i\geqq0)$, ($x,c$は$n次元ベクトル$)
\end{itembox}

上記の標準形に対して、$rankA = m$とする。($m>n$)\\
この時、係数行列$A$の中には線型独立な列ベクトルが$m$本存在する。($rankA=m$なので当然。)\\
それらを並べて正則行列を作ったとき、この行列を$\textbf{基底行列}$と呼ぶ。それらに対応する変数を基底変数、それ以外の変数を非基底変数と呼ぶ。\\
$A= \begin{pmatrix}
B \quad N\\
\end{pmatrix}
$, \quad
$x= \begin{pmatrix}
x_B\\
x_N
\end{pmatrix}
$\\
基底行列を$B$,非基底行列を$N$,基底変数からなるベクトルを$x_B$, 非基底変数のベクトルを$x_N$とおくと、上記のように$A,x$を表現することで $ Ax=b$を書き換えることができる。\\
$Ax = b$ は$B x_B + N x_N = b$(ただ行列の掛け算をしただけ)\\
非基底変数ベクトル$x_N$を0とすると$x_B = B^{-1} b$と定まる。\\
この時の解、つまり
$x= \begin{pmatrix}
B^{-1} b \\
0 \\
\end{pmatrix}
$
のことを$\textbf{基底解}$と呼ぶ。\\
基底解のすべての変数が非負の時、その解は実行可能基底解である。\\
また、実行可能基底解の中でちょうどm個の基底変数の値が正(つまり0でない)とき$\textbf{非退化}$と呼ぶ。具体例を見たほうがわかりやすいので、具体例を見てみよう。\\
$A= \begin{pmatrix}
5 \quad 2 \quad 4 \quad 7 \quad 1\\
1 \quad 2 \quad 4 \quad 3 \quad -3
\end{pmatrix}
$, \quad
$x= \begin{pmatrix}
x_1 \\
x_2 \\
x_3 \\
x_4 \\
x_5 \\
\end{pmatrix}
$,\quad
$b= \begin{pmatrix}
14 \\
6 \\
\end{pmatrix}
$ \\
この時、$rankA = 2$である。\\
基底行列を
$B= \begin{pmatrix}
5 \quad 2 \\
1 \quad 2 \\
\end{pmatrix}
$にしてみる。勿論、これ以外にも基底行列の選び方は様々あるが、試しにこれを選んでいる。\\
Bを基底行列に選ぶと、それに対応する
$x_B= \begin{pmatrix}
x_1 \\
x_2 \\
\end{pmatrix}
$が基底変数になる。非基底変数は0とみなすので$B x_B = b$となり、$x_B= \begin{pmatrix}
2 \\
2 \\
\end{pmatrix}
$となる。(計算略)\\
ここで、基底解を見てみると
$x= \begin{pmatrix}
2 \\
2 \\
0 \\
0 \\
0 \\
\end{pmatrix}
$になっている。0ではない数字が2つあるので、この解は非退化基底解になる。\\


\subsection*{3-4.線型計画法の基本定理}

\begin{itembox}[l]{線型計画問題の基本定理}
(1) 実行可能解が存在するとすると、実行可能基底解も存在する。\\
(2) 最適解が存在するならば、実行可能基底解の中にある。
\end{itembox}

実行可能解を
$X= \begin{pmatrix}
X_1 \\
X_2 \\
\vdots\\
X_n \\
\end{pmatrix}
$,それに対応するAの列ベクトルを
$A= \begin{pmatrix}
A_1 \\
A_2 \\
\vdots\\
A_n \\
\end{pmatrix}
$とすると、$AX=b$より$A_1 X_1 + A_2 X_2 \cdots A_n X_n = b$が成立。\\
Xの成分のうち、0ではない(正の値である)ものの個数で場合分けして考える。\\
(i)0でないXの成分が存在する場合\\
0でないXの成分の個数をKとおく。K個以外のXの成分は全て0であるので、最初のK個の成分が正である、という仮定をおいても一般性を失わない。\\
$Ax = b$より、$X_1 A_1 + X_2 A_2 + \cdots +X_K A_K = b $が成立
\noindent
$A_1~A_K$が線型独立であり、かつK=mだった場合、Xは非退化実行可能基底解となる。
(この時の基底は$X_1~X_m$)\\
$K < m$の時、選んでいないAの成分の中から、うまい具合に列を選んでみると、$A_1,A_2, \cdots A_m$が線型独立になるような組み合わせを実現することができる。この時Xは($X_1,X_2, \cdots  X_m$)を基底とする退化した実行可能基底解となる。
($A_1~A_K$は,$rankA = m$を満たす行列Aの中から線型独立なものをK個選んだものなので上記のような都合のいい選び方ができる。ちなみに$K=0$の時も、この方法に帰着できる)

(ii) $A_1,A_2, \cdots A_n$が線型従属であるとき\\
$A_1,A_2, \cdots A_k$が線型従属であるとき、少なくとも一つが0ではないスカラー$a_1,a_2, \cdots , a_K$が存在し、$a_1 A_1 + a_2 A_2 + \cdots a_k A_K=0$が成立する。この式を★とする\\
(線型従属なので、$a_s = t a_u$となる組み合わせが存在するということ。それ以外を0にすれば上式が成立する)\\
任意の定数$Z$を★に掛けて、$A_1 X_1 + A_2 X_2 \cdots A_n X_n = b$から引き算すると、\\
$(X_1 - Z a_1)A_1 + (X_2 - Z a_2)A_2 + \cdots + (X_K - Z a_K)A_K = b$が成立する。$A_1~A_K$のうち、少なくとも1つは正なので$Z$を増加させると、$X_i -Z a_i$は小さくなっていく。\\
$(X_t - Z a_t)$が負になってしまうと制約条件に違反してしまうので、すべての$(X_t - Z a_t)$が0以上になるように$Z$を定めればいい。つまり$Z = min \dfrac{X_i}{a_i}$ とすれば万事解決。$X_i - Z a_i=0$となる場所が必ず出現するため、この操作を行えば1要素を消す(=0にする)ことができる
この操作を線型従属の状態が消えるまで繰り返せば、先述した方法に帰着できる。\\
(2)最適解が存在するならば、実行可能基底解の中にある
最適解の1つを
$X= \begin{pmatrix}
X_1 \\
X_2 \\
\vdots\\
X_n \\
\end{pmatrix}
$とし、それに対応するAの列ベクトルを
$A= \begin{pmatrix}
A_1 \quad A_2 \quad \cdots A_n
\end{pmatrix}
$
とする。X=0の時は(1)のK=0の時と同じなので割愛。\\
以下、$X \neq 0$として考える。\\
最初のK個が正として仮定しても一般性を失わない。
$A_1 X_1 + A_2 X_2 \cdots A_K X_K = b$が成立する。
$A_1 + A_2 \cdots A_K = b$が線型独立であるとき、(1)の証明の時と同じようにすれば丸く収まる。(つまり最適解Xが実行可能基底解になる)\\
次に線型従属の場合を考える。線型従属の時、少なくとも一つが0ではないスカラー$a_1,a_2, \cdots , a_K$が存在し、$a_1 A_1 + a_2 A_2 + \cdots a_k A_K=0$が成立する。\\
$|\delta|$ が十分に小さいとき、
$\bar{X}= \begin{pmatrix}
\bar{x}_1 + \delta a_1 \\
\bar{x}_2 + \delta a_2 \\
\cdots \\
\bar{x}_K + \delta a_K\\
0 \\
0 \\
\cdots \\
0
\end{pmatrix}
$\\
は実行可能解になる。($|\delta|$ が十分に小さいときには、$\bar{X}$は殆どXと等しいので、制約範囲内に収まる)\\



$c^{T} X \leq c^{T} \bar{X} = \sum\limits_{i=1}^K c_i (\bar{X} + \delta a_i ) = c^{T} X+ \delta \sum\limits_{i=1}^K c_i a_i $\\
↑最適解ではないので、コストが少し大きくなってしまう\\
$0 \leq \delta \sum\limits_{i=1}^K c_i a_i$が成立する。この式は$\delta$の正負に関わらず成立するので、$\sum\limits_{i=1}^K c_i a_i = 0$\\
後の流れはさっきやったやつ($Z$かけて引くやつ)にそっくりなので省略。\\
すると、同じように実行可能基底解を得ることができる。
この基底解を代入して計算すると,任意の$\epsilon$をおいて\\
$\sum\limits_{i=1}^K c_i (\bar{X}_i - \epsilon a_i) = c^{T} \bar{X} - \epsilon \sum\limits_{i=1}^K c_i a_i$\\
$\sum\limits_{i=1}^K c_i a_i = 0$より
$c^{T} \bar{X} - \epsilon \sum\limits_{i=1}^K c_i a_i = c^{T} \bar{X}$。
これは最適解の時の解と等しい。\\
以上より、実行可能基底解の中に最適解がある。


これらの基本定理より、最適解を調べる際には実行可能基底解を調べれば大丈夫、という事がわかった。\\
実行可能基底解の個数は${}_n C_m$通りしかないので、有限回の操作を行えば最適解を求めることができる。それを行っているのが次に紹介する単体法である。




\subsection*{3-5.単体法}
行列Aの最初の方のm本の列ベクトルが線型独立であると仮定しても一般性を失わない。\\
基底行列をB,基底変数を$x_B$,非基底行列をN,非基底変数を$x_N$とおくと,制約条件は$B x_B + N x_N =b$\\
左から$B^{-1}$を掛けると$x_B + B^{-1} N x_N = B^{-1} b$\\
これを目的関数に代入すると、$w = c_B^{T} B^{-1}b + (c_N - (B^{-1} N)^{T} c_B)^{T} x_N$\\
$Y = (B^{-1})^{T} c_B$を用いると、$w = c_B^{T} B^{-1}b + (c_N - N^{T} y)^{T} x_N $と表現できる。\\
$\bar{N}= B^{-1} N $, \quad 
$\bar{c} = c_N - N^{T} y$, \quad
$-w + \bar{c_N}^{T} x_N = - \bar{w}$, \\
$\bar{b} = x_B + \bar{N} x_N$としたとき、$\bar{b} \leq 0$なら実行可能基底形式である。($x_N=0$とすると$x_B = b$という関係式が求められる)\\
では早速単体法をやってみよう!\\
まずは実行可能基底解を決定する。
今回は
$x= \begin{pmatrix}
x_B\\
x_N
\end{pmatrix}
$
=
$\begin{pmatrix}
\bar{b}\\
0
\end{pmatrix}
$
を仮に定めておく。これももとに目的関数の値が小さくなるように解を更新していくのが単体法のやり方である。\\
$w - \bar{c_N}^{T} x_N =  \bar{w}$より、$w = \bar{c_N}^{T} x_N +  \bar{w}$. このwを最小にする解が最適解。

$\bar{c_N} \geq 0$の時、$w = \bar{c_N}^{T} x_N +  \bar{w}$より、$w \geq \bar{w}$となってしまうので、これ以上小さくすることができない。よってこれが最適解。\\


一方、$\bar{c_q} \leq \< 0$なるqが存在してしまうと、$w = \bar{w} + \bar{c_q} x_q +(\bar{c_q} x_q以外のやつの和)$より、非基底変数$x_q$の値を増加させると、目的関数地を減少させることができる。\\
つまり、まだまだ最適化の余地があるという事になる。減少させる際にはどの添え字$q$を増加させてもいいが、基本的には$\bar{c_q}$が最も小さいものを選べば大丈夫!

そのようにして$q$を選び、$x_q$以外の非基底変数を0にすると
$x_B = \bar{b}-\sum\limits_{i=m+1}^n x_i \bar{a_i} $, \quad
$-w =  - \bar{W} -\sum\limits_{i=m+a}^n  \bar{c_i} x_i $
は以下のように変形できる。\\\\

\large
$x_B = \bar{b} + x_q (-\bar{a_q}) $, \quad
$w = \bar{w} + \bar{c_q} x_q$
\normalsize

ここで$\bar{a_q}$の符号について考えてみる。\\
1. $\bar{a_q}$が負の時\\
$x_B = \bar{b} + x_q (-\bar{a_q}) $より、$x_B \geq 0$という条件は必ず満たすことができる。$\bar{x_q}$を非常に大きくすれば$\bar{c_q} \leq 0$より
いくらでも$w$を小さくすることができる。\\
そのため、この場合は下に有界ではなくなってしまうのでOUT.\\
\noindent \\
2.それ以外の時\\
先ほどの場合のように下に有界ではない、みたいな状況は起こらないので、$\bar{x_B} \geq 0$の条件を遵守して$x_q$を選べば大丈夫。\\
$\bar{b} + x_q (-\bar{a_q}) =0 $
つまり、
$x = \dfrac{\bar{b}}{\bar{a_q}}$
となる場合である。\\
一点だけ注意点がある。
$\dfrac{\bar{b}}{\bar{a_q}}$を選ぶときには、一番小さいものを選ぶ必要がある。
なぜなら、すべての$x$について非負制約を満たす必要があり、変なものを選んでしまうと、府になってしまうものが現れ不適になってしまうからである。\\
ここで選んだ基底変数$x_p$を基底変数から外し、
「$w = \bar{w} + \bar{c_q} x_q +(\bar{c_q} x_q以外のやつの和)$」
で出てきた非基底変数$x_q$を基底変数の中に組み込む。\\
すると、$x_q$の次に最小だったものが$\bar{c_q}$として採択されて、同じような手順を繰り返すことができる。(同じ手順をしなくてももっと簡単な方法で求めることもできる)








\end{document}
