\RequirePackage{plautopatch}
\RequirePackage[l2tabu, orthodox]{nag}

\documentclass[platex,dvipdfmx]{jlreq}			% for platex
% \documentclass[uplatex,dvipdfmx]{jlreq}		% for uplatex
\usepackage{graphicx}
\usepackage{bxtexlogo}

\title{最適化ゼミ}

\author{T-N}
\date{\today}
\begin{document}
\maketitle
\section{2段階法}

前のセクションで述べた単体法は、実行可能基底解が存在するという仮定の下、最適化の作業を進めていったが、実際の最適化問題では「そもそも制約条件が矛盾していて、実行可能解が存在しない」「実行可能基底解を標準形から容易に見つけることができない」ような事例に遭遇することがある。





\end{document}
